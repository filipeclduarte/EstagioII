\documentclass[]{article}
\usepackage{lmodern}
\usepackage{amssymb,amsmath}
\usepackage{ifxetex,ifluatex}
\usepackage{fixltx2e} % provides \textsubscript
\ifnum 0\ifxetex 1\fi\ifluatex 1\fi=0 % if pdftex
  \usepackage[T1]{fontenc}
  \usepackage[utf8]{inputenc}
\else % if luatex or xelatex
  \ifxetex
    \usepackage{mathspec}
  \else
    \usepackage{fontspec}
  \fi
  \defaultfontfeatures{Ligatures=TeX,Scale=MatchLowercase}
\fi
% use upquote if available, for straight quotes in verbatim environments
\IfFileExists{upquote.sty}{\usepackage{upquote}}{}
% use microtype if available
\IfFileExists{microtype.sty}{%
\usepackage{microtype}
\UseMicrotypeSet[protrusion]{basicmath} % disable protrusion for tt fonts
}{}
\usepackage[margin=1in]{geometry}
\usepackage{hyperref}
\hypersetup{unicode=true,
            pdftitle={Aula 3 - Estágio Supervisionado II},
            pdfauthor={Filipe Duarte},
            pdfborder={0 0 0},
            breaklinks=true}
\urlstyle{same}  % don't use monospace font for urls
\usepackage{color}
\usepackage{fancyvrb}
\newcommand{\VerbBar}{|}
\newcommand{\VERB}{\Verb[commandchars=\\\{\}]}
\DefineVerbatimEnvironment{Highlighting}{Verbatim}{commandchars=\\\{\}}
% Add ',fontsize=\small' for more characters per line
\usepackage{framed}
\definecolor{shadecolor}{RGB}{248,248,248}
\newenvironment{Shaded}{\begin{snugshade}}{\end{snugshade}}
\newcommand{\AlertTok}[1]{\textcolor[rgb]{0.94,0.16,0.16}{#1}}
\newcommand{\AnnotationTok}[1]{\textcolor[rgb]{0.56,0.35,0.01}{\textbf{\textit{#1}}}}
\newcommand{\AttributeTok}[1]{\textcolor[rgb]{0.77,0.63,0.00}{#1}}
\newcommand{\BaseNTok}[1]{\textcolor[rgb]{0.00,0.00,0.81}{#1}}
\newcommand{\BuiltInTok}[1]{#1}
\newcommand{\CharTok}[1]{\textcolor[rgb]{0.31,0.60,0.02}{#1}}
\newcommand{\CommentTok}[1]{\textcolor[rgb]{0.56,0.35,0.01}{\textit{#1}}}
\newcommand{\CommentVarTok}[1]{\textcolor[rgb]{0.56,0.35,0.01}{\textbf{\textit{#1}}}}
\newcommand{\ConstantTok}[1]{\textcolor[rgb]{0.00,0.00,0.00}{#1}}
\newcommand{\ControlFlowTok}[1]{\textcolor[rgb]{0.13,0.29,0.53}{\textbf{#1}}}
\newcommand{\DataTypeTok}[1]{\textcolor[rgb]{0.13,0.29,0.53}{#1}}
\newcommand{\DecValTok}[1]{\textcolor[rgb]{0.00,0.00,0.81}{#1}}
\newcommand{\DocumentationTok}[1]{\textcolor[rgb]{0.56,0.35,0.01}{\textbf{\textit{#1}}}}
\newcommand{\ErrorTok}[1]{\textcolor[rgb]{0.64,0.00,0.00}{\textbf{#1}}}
\newcommand{\ExtensionTok}[1]{#1}
\newcommand{\FloatTok}[1]{\textcolor[rgb]{0.00,0.00,0.81}{#1}}
\newcommand{\FunctionTok}[1]{\textcolor[rgb]{0.00,0.00,0.00}{#1}}
\newcommand{\ImportTok}[1]{#1}
\newcommand{\InformationTok}[1]{\textcolor[rgb]{0.56,0.35,0.01}{\textbf{\textit{#1}}}}
\newcommand{\KeywordTok}[1]{\textcolor[rgb]{0.13,0.29,0.53}{\textbf{#1}}}
\newcommand{\NormalTok}[1]{#1}
\newcommand{\OperatorTok}[1]{\textcolor[rgb]{0.81,0.36,0.00}{\textbf{#1}}}
\newcommand{\OtherTok}[1]{\textcolor[rgb]{0.56,0.35,0.01}{#1}}
\newcommand{\PreprocessorTok}[1]{\textcolor[rgb]{0.56,0.35,0.01}{\textit{#1}}}
\newcommand{\RegionMarkerTok}[1]{#1}
\newcommand{\SpecialCharTok}[1]{\textcolor[rgb]{0.00,0.00,0.00}{#1}}
\newcommand{\SpecialStringTok}[1]{\textcolor[rgb]{0.31,0.60,0.02}{#1}}
\newcommand{\StringTok}[1]{\textcolor[rgb]{0.31,0.60,0.02}{#1}}
\newcommand{\VariableTok}[1]{\textcolor[rgb]{0.00,0.00,0.00}{#1}}
\newcommand{\VerbatimStringTok}[1]{\textcolor[rgb]{0.31,0.60,0.02}{#1}}
\newcommand{\WarningTok}[1]{\textcolor[rgb]{0.56,0.35,0.01}{\textbf{\textit{#1}}}}
\usepackage{graphicx,grffile}
\makeatletter
\def\maxwidth{\ifdim\Gin@nat@width>\linewidth\linewidth\else\Gin@nat@width\fi}
\def\maxheight{\ifdim\Gin@nat@height>\textheight\textheight\else\Gin@nat@height\fi}
\makeatother
% Scale images if necessary, so that they will not overflow the page
% margins by default, and it is still possible to overwrite the defaults
% using explicit options in \includegraphics[width, height, ...]{}
\setkeys{Gin}{width=\maxwidth,height=\maxheight,keepaspectratio}
\IfFileExists{parskip.sty}{%
\usepackage{parskip}
}{% else
\setlength{\parindent}{0pt}
\setlength{\parskip}{6pt plus 2pt minus 1pt}
}
\setlength{\emergencystretch}{3em}  % prevent overfull lines
\providecommand{\tightlist}{%
  \setlength{\itemsep}{0pt}\setlength{\parskip}{0pt}}
\setcounter{secnumdepth}{0}
% Redefines (sub)paragraphs to behave more like sections
\ifx\paragraph\undefined\else
\let\oldparagraph\paragraph
\renewcommand{\paragraph}[1]{\oldparagraph{#1}\mbox{}}
\fi
\ifx\subparagraph\undefined\else
\let\oldsubparagraph\subparagraph
\renewcommand{\subparagraph}[1]{\oldsubparagraph{#1}\mbox{}}
\fi

%%% Use protect on footnotes to avoid problems with footnotes in titles
\let\rmarkdownfootnote\footnote%
\def\footnote{\protect\rmarkdownfootnote}

%%% Change title format to be more compact
\usepackage{titling}

% Create subtitle command for use in maketitle
\providecommand{\subtitle}[1]{
  \posttitle{
    \begin{center}\large#1\end{center}
    }
}

\setlength{\droptitle}{-2em}

  \title{Aula 3 - Estágio Supervisionado II}
    \pretitle{\vspace{\droptitle}\centering\huge}
  \posttitle{\par}
    \author{Filipe Duarte}
    \preauthor{\centering\large\emph}
  \postauthor{\par}
      \predate{\centering\large\emph}
  \postdate{\par}
    \date{8/14/2019}


\begin{document}
\maketitle

\hypertarget{aula-3}{%
\section{Aula 3}\label{aula-3}}

Nesta aula veremos como estimar os parâmetros pelo método da máxima
verossimilhança para algumas distribuições e, em seguida, como
selecionar qual o modelo com o melhor ajuste.

\hypertarget{metodo-da-maxima-verossimilhanca}{%
\subsection{Método da Máxima
Verossimilhança}\label{metodo-da-maxima-verossimilhanca}}

O método da máxima verossimilhança

\hypertarget{fitditsrplus}{%
\section{fitditsrplus}\label{fitditsrplus}}

Carregando biblioteca:

\begin{Shaded}
\begin{Highlighting}[]
\KeywordTok{library}\NormalTok{(fitdistrplus)}
\end{Highlighting}
\end{Shaded}

\begin{verbatim}
## Loading required package: MASS
\end{verbatim}

\begin{verbatim}
## Loading required package: survival
\end{verbatim}

\begin{verbatim}
## Loading required package: npsurv
\end{verbatim}

\begin{verbatim}
## Loading required package: lsei
\end{verbatim}

Carregando dados:

\begin{Shaded}
\begin{Highlighting}[]
\NormalTok{sinistro <-}\StringTok{ }\KeywordTok{read.csv}\NormalTok{(}\StringTok{"sinistros.csv"}\NormalTok{, }\DataTypeTok{header =} \OtherTok{TRUE}\NormalTok{, }\DataTypeTok{sep=}\StringTok{","}\NormalTok{)}
\end{Highlighting}
\end{Shaded}

Visualizando os dados:

\begin{Shaded}
\begin{Highlighting}[]
\KeywordTok{head}\NormalTok{(sinistro)}
\end{Highlighting}
\end{Shaded}

\begin{verbatim}
##   X          x
## 1 1  172.28719
## 2 2  242.73857
## 3 3  445.94377
## 4 4  815.13975
## 5 5 1302.07868
## 6 6   90.00823
\end{verbatim}

Criando um vetor apenas com os valores dos sinistros:

\begin{Shaded}
\begin{Highlighting}[]
\NormalTok{sinistro <-}\StringTok{ }\NormalTok{sinistro}\OperatorTok{$}\NormalTok{x}
\end{Highlighting}
\end{Shaded}

Estatísticas descritivas:

\begin{Shaded}
\begin{Highlighting}[]
\CommentTok{# summary}
\KeywordTok{summary}\NormalTok{(sinistro)}
\end{Highlighting}
\end{Shaded}

\begin{verbatim}
##    Min. 1st Qu.  Median    Mean 3rd Qu.    Max. 
##   12.11  216.65  441.26  535.30  796.66 1818.77
\end{verbatim}

\begin{Shaded}
\begin{Highlighting}[]
\CommentTok{# desvio padrão}
\KeywordTok{sd}\NormalTok{(sinistro)}
\end{Highlighting}
\end{Shaded}

\begin{verbatim}
## [1] 410.4036
\end{verbatim}

Histograma

\begin{Shaded}
\begin{Highlighting}[]
\CommentTok{# histograma}
\KeywordTok{hist}\NormalTok{(sinistro)}
\end{Highlighting}
\end{Shaded}

\includegraphics{Aula_3_-_Estágio_Supervisionado_II_files/figure-latex/unnamed-chunk-6-1.pdf}

\hypertarget{ajuste-das-distribuicoes}{%
\section{Ajuste das distribuições}\label{ajuste-das-distribuicoes}}

\begin{Shaded}
\begin{Highlighting}[]
\KeywordTok{descdist}\NormalTok{(sinistro)}
\end{Highlighting}
\end{Shaded}

\includegraphics{Aula_3_-_Estágio_Supervisionado_II_files/figure-latex/unnamed-chunk-7-1.pdf}

\begin{verbatim}
## summary statistics
## ------
## min:  12.11001   max:  1818.766 
## median:  441.2607 
## mean:  535.2991 
## estimated sd:  410.4036 
## estimated skewness:  1.039267 
## estimated kurtosis:  3.838951
\end{verbatim}

Vamos testar para gamma, weibull e lognormal:

\begin{Shaded}
\begin{Highlighting}[]
\NormalTok{fit <-}\StringTok{ }\KeywordTok{fitdist}\NormalTok{(sinistro, }\StringTok{"gamma"}\NormalTok{)}
\KeywordTok{fitdist}\NormalTok{(sinistro, }\StringTok{"weibull"}\NormalTok{)}
\end{Highlighting}
\end{Shaded}

\begin{verbatim}
## Fitting of the distribution ' weibull ' by maximum likelihood 
## Parameters:
##         estimate Std. Error
## shape   1.281123  0.1020256
## scale 577.074872 47.3792335
\end{verbatim}

\begin{Shaded}
\begin{Highlighting}[]
\KeywordTok{fitdist}\NormalTok{(sinistro, }\StringTok{"lnorm"}\NormalTok{)}
\end{Highlighting}
\end{Shaded}

\begin{verbatim}
## Fitting of the distribution ' lnorm ' by maximum likelihood 
## Parameters:
##         estimate Std. Error
## meanlog 5.896328 0.10208724
## sdlog   1.020872 0.07218627
\end{verbatim}

Vamos plotar o histograma com a curva gerada a partir da estimação:

\begin{Shaded}
\begin{Highlighting}[]
\KeywordTok{hist}\NormalTok{(sinistro, }\DataTypeTok{pch=}\DecValTok{20}\NormalTok{, }\DataTypeTok{breaks =} \DecValTok{20}\NormalTok{, }\DataTypeTok{prob =} \OtherTok{TRUE}\NormalTok{, }\DataTypeTok{main =} \StringTok{""}\NormalTok{)}
\KeywordTok{curve}\NormalTok{(}\KeywordTok{dgamma}\NormalTok{(x, fit}\OperatorTok{$}\NormalTok{estimate[}\DecValTok{1}\NormalTok{], fit}\OperatorTok{$}\NormalTok{estimate[}\DecValTok{2}\NormalTok{]), }\DataTypeTok{add =} \OtherTok{TRUE}\NormalTok{, }\DataTypeTok{col =} \StringTok{"red"}\NormalTok{)}
\end{Highlighting}
\end{Shaded}

\includegraphics{Aula_3_-_Estágio_Supervisionado_II_files/figure-latex/unnamed-chunk-9-1.pdf}

Vamos fazer o histograma e a distribuição acumulada:

\begin{Shaded}
\begin{Highlighting}[]
\KeywordTok{plotdist}\NormalTok{(sinistro, }\DataTypeTok{histo =} \OtherTok{TRUE}\NormalTok{, }\DataTypeTok{demp =} \OtherTok{TRUE}\NormalTok{)}
\end{Highlighting}
\end{Shaded}

\includegraphics{Aula_3_-_Estágio_Supervisionado_II_files/figure-latex/unnamed-chunk-10-1.pdf}

Agora vamos estimar novamente para criar as 3 funções

\begin{Shaded}
\begin{Highlighting}[]
\NormalTok{fit_w  <-}\StringTok{ }\KeywordTok{fitdist}\NormalTok{(sinistro, }\StringTok{"weibull"}\NormalTok{)}
\NormalTok{fit_g  <-}\StringTok{ }\KeywordTok{fitdist}\NormalTok{(sinistro, }\StringTok{"gamma"}\NormalTok{)}
\NormalTok{fit_ln <-}\StringTok{ }\KeywordTok{fitdist}\NormalTok{(sinistro, }\StringTok{"lnorm"}\NormalTok{)}
\end{Highlighting}
\end{Shaded}

\begin{Shaded}
\begin{Highlighting}[]
\KeywordTok{summary}\NormalTok{(fit_w)}
\end{Highlighting}
\end{Shaded}

\begin{verbatim}
## Fitting of the distribution ' weibull ' by maximum likelihood 
## Parameters : 
##         estimate Std. Error
## shape   1.281123  0.1020256
## scale 577.074872 47.3792335
## Loglikelihood:  -723.9747   AIC:  1451.949   BIC:  1457.16 
## Correlation matrix:
##           shape     scale
## shape 1.0000000 0.3092276
## scale 0.3092276 1.0000000
\end{verbatim}

\begin{Shaded}
\begin{Highlighting}[]
\KeywordTok{summary}\NormalTok{(fit_g)}
\end{Highlighting}
\end{Shaded}

\begin{verbatim}
## Fitting of the distribution ' gamma ' by maximum likelihood 
## Parameters : 
##          estimate   Std. Error
## shape 1.437275405 0.1469102032
## rate  0.002684913 0.0002710074
## Loglikelihood:  -724.6539   AIC:  1453.308   BIC:  1458.518 
## Correlation matrix:
##           shape      rate
## shape 1.0000000 0.7292785
## rate  0.7292785 1.0000000
\end{verbatim}

\begin{Shaded}
\begin{Highlighting}[]
\KeywordTok{summary}\NormalTok{(fit_ln)}
\end{Highlighting}
\end{Shaded}

\begin{verbatim}
## Fitting of the distribution ' lnorm ' by maximum likelihood 
## Parameters : 
##         estimate Std. Error
## meanlog 5.896328 0.10208724
## sdlog   1.020872 0.07218627
## Loglikelihood:  -733.5924   AIC:  1471.185   BIC:  1476.395 
## Correlation matrix:
##         meanlog sdlog
## meanlog       1     0
## sdlog         0     1
\end{verbatim}

\begin{Shaded}
\begin{Highlighting}[]
\KeywordTok{par}\NormalTok{(}\DataTypeTok{mfrow=}\KeywordTok{c}\NormalTok{(}\DecValTok{2}\NormalTok{,}\DecValTok{2}\NormalTok{))}
\NormalTok{plot.legend <-}\StringTok{ }\KeywordTok{c}\NormalTok{(}\StringTok{"Weibull"}\NormalTok{, }\StringTok{"lognormal"}\NormalTok{, }\StringTok{"gamma"}\NormalTok{)}
\KeywordTok{denscomp}\NormalTok{(}\KeywordTok{list}\NormalTok{(fit_w, fit_g, fit_ln), }\DataTypeTok{legendtext =}\NormalTok{ plot.legend)}
\KeywordTok{cdfcomp}\NormalTok{ (}\KeywordTok{list}\NormalTok{(fit_w, fit_g, fit_ln), }\DataTypeTok{legendtext =}\NormalTok{ plot.legend)}
\KeywordTok{qqcomp}\NormalTok{  (}\KeywordTok{list}\NormalTok{(fit_w, fit_g, fit_ln), }\DataTypeTok{legendtext =}\NormalTok{ plot.legend)}
\KeywordTok{ppcomp}\NormalTok{  (}\KeywordTok{list}\NormalTok{(fit_w, fit_g, fit_ln), }\DataTypeTok{legendtext =}\NormalTok{ plot.legend)}
\end{Highlighting}
\end{Shaded}

\includegraphics{Aula_3_-_Estágio_Supervisionado_II_files/figure-latex/unnamed-chunk-12-1.pdf}

Testando outras distribuições:

Carregar outro pacote:

\begin{Shaded}
\begin{Highlighting}[]
\KeywordTok{library}\NormalTok{(actuar)}
\end{Highlighting}
\end{Shaded}

\begin{verbatim}
## 
## Attaching package: 'actuar'
\end{verbatim}

\begin{verbatim}
## The following object is masked from 'package:grDevices':
## 
##     cm
\end{verbatim}

Testando a loglogistica e pareto:

\begin{Shaded}
\begin{Highlighting}[]
\NormalTok{fit_ll <-}\StringTok{ }\KeywordTok{fitdist}\NormalTok{(sinistro, }\StringTok{"llogis"}\NormalTok{, }\DataTypeTok{start =} \KeywordTok{list}\NormalTok{(}\DataTypeTok{shape =} \DecValTok{1}\NormalTok{, }\DataTypeTok{scale =} \DecValTok{500}\NormalTok{))}
\NormalTok{fit_P  <-}\StringTok{ }\KeywordTok{fitdist}\NormalTok{(sinistro, }\StringTok{"pareto"}\NormalTok{, }\DataTypeTok{start =} \KeywordTok{list}\NormalTok{(}\DataTypeTok{shape =} \DecValTok{1}\NormalTok{, }\DataTypeTok{scale =} \DecValTok{500}\NormalTok{))}
\end{Highlighting}
\end{Shaded}

\begin{verbatim}
## Warning in sqrt(diag(varcovar)): NaNs produced
\end{verbatim}

\begin{verbatim}
## Warning in sqrt(1/diag(V)): NaNs produced
\end{verbatim}

\begin{verbatim}
## Warning in cov2cor(varcovar): diag(.) had 0 or NA entries; non-finite
## result is doubtful
\end{verbatim}

\begin{Shaded}
\begin{Highlighting}[]
\KeywordTok{cdfcomp}\NormalTok{(}\KeywordTok{list}\NormalTok{(fit_ln, fit_ll, fit_P), }\DataTypeTok{xlogscale =} \OtherTok{TRUE}\NormalTok{, }\DataTypeTok{ylogscale =} \OtherTok{TRUE}\NormalTok{,}
        \DataTypeTok{legendtext =} \KeywordTok{c}\NormalTok{(}\StringTok{"lognormal"}\NormalTok{, }\StringTok{"loglogistic"}\NormalTok{, }\StringTok{"Pareto"}\NormalTok{), }\DataTypeTok{lwd=}\DecValTok{2}\NormalTok{)}
\end{Highlighting}
\end{Shaded}

\includegraphics{Aula_3_-_Estágio_Supervisionado_II_files/figure-latex/unnamed-chunk-14-1.pdf}

\begin{Shaded}
\begin{Highlighting}[]
\KeywordTok{gofstat}\NormalTok{(}\KeywordTok{list}\NormalTok{(fit_ln, fit_ll, fit_P), }\DataTypeTok{fitnames =} \KeywordTok{c}\NormalTok{(}\StringTok{"lnorm"}\NormalTok{, }\StringTok{"llogis"}\NormalTok{, }\StringTok{"Pareto"}\NormalTok{))}
\end{Highlighting}
\end{Shaded}

\begin{verbatim}
## Goodness-of-fit statistics
##                                  lnorm     llogis    Pareto
## Kolmogorov-Smirnov statistic 0.1202205 0.08212615 0.1157534
## Cramer-von Mises statistic   0.3234141 0.16308113 0.3195346
## Anderson-Darling statistic   1.8945669 1.41603928 1.7619729
## 
## Goodness-of-fit criteria
##                                   lnorm   llogis   Pareto
## Akaike's Information Criterion 1471.185 1469.141 1460.565
## Bayesian Information Criterion 1476.395 1474.351 1465.775
\end{verbatim}


\end{document}
